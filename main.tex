\documentclass[a4paper, 12pt, sans]{moderncv}

\usepackage[a4paper, top=0.5cm, bottom=1.5cm, left=1.75cm, right=1.75cm]{geometry}
\usepackage[T1]{fontenc}
\usepackage[utf8]{inputenc}

\usepackage[english]{babel}

\usepackage{mathtext}
\usepackage{mathtools}
\usepackage{mathrsfs}
\usepackage{amsmath,amssymb}
\usepackage{float}

\usepackage{indentfirst}
\usepackage{longtable}


\moderncvstyle{classic}
\moderncvcolor{burgundy}
\setlength{\hintscolumnwidth}{3.0cm}

\firstname{Kirill}
\familyname{Grinko}
\mobile{+7 916 825 87 57}
\email{k.a.grinko@gmail.com}
\social[telegram]{pr1kol2}
\social[github]{pr1kol2005}

\begin{document}

\makecvtitle


\section{Personal}

\cvitem{Hard skills}{C++, Algorithms \&  data structures, Concurrency, C, Assembly x86-64, Python, LaTeX, Git, CMake, Gitlab CI/CD, GoogleTest, Bash, Docker, Qt, SFML.}

\cvitem{Soft skills}{Hard-working, Quick-learning, Organised, Outgoing and collaborative.}

\cvitem{Languages}{English (B2), Chinese (A1), Russian (native speaker).}

\cvitem{Hobbies}{Calisthenics, skiing, cycling, piano.}


\section{Projects}

\cventry{Spring 2025}{Concurrency course homework}{}{Used tools: C++}{}{Implemented various synchronization primitives using atomic operations only. Built a thread pool and stackful coroutines. Combined them to create fibers (user-space cooperative threads) and implemented synchronization primitives for them. Developed functional combinators for working with futures (representing values computed by asynchronous operations). Implemented a lock-free data structures (atomic shared\_ptr, stack, queue) using the hazard pointers scheme. Upcoming: work-stealing thread pool, channels for fibers, laziness optimization for futures.}

\cventry{Fall 2024 -- Spring 2025}{C++ course homework}{\href{https://github.com/pr1kol2005/cpp-3-4-term}{Github page (clickable)}}{Used tools: C++}{}{Implemented template allocator-aware data structures (unordered\_map, list, smart pointers, strategy-based array, matrix), type-erased configuration system with vtable, compile-time 8-puzzle solver, JSON converter, geometry primitives, big\_integer.}

\cventry{Fall 2024}{Graphing Calculator}{\href{https://github.com/pr1kol2005/graphing-calculator}{Github page (clickable)}}{Used tools: C++, SFML, CMake}{}{A graphing calculator and plotter application.}

\cventry{Fall 2024 -- Spring 2025}{Algorithms and data structures course homework}{\href{https://github.com/pr1kol2005/algorithms-collection.git}{Github page (clickable)}}{Used tools: C++}{}{Implemented solutions to competitive programming problems covering fundamental algorithms and data structures, dynamic programming techniques, graph algorithms, algorithms on strings, and number theory algorithms.}

\cventry{Spring 2024}{Box with molecules}{\href{https://github.com/pr1kol2005/box-with-molecules}{Github page (clickable)}}{Used tools: C++, Qt, CMake}{}{A simulation of an ideal gas in an enclosed space, including a small research component to test the validity of the Maxwell distribution.}

\cventry{Fall 2023 -- Spring 2024}{Physics Laboratory Works}{\href{https://github.com/pr1kol2005/labs-mipt.git}{Github page (clickable)}}{Used tools: LaTeX, Python}{}{A collection of completed laboratory works in physics, including theoretical calculations, experimental data analysis, and visualizations using Python.}

\cventry{Fall 2023}{MBTI test}{\href{https://github.com/pr1kol2005/test-mbti}{Github page (clickable)}}{Used tools: Python, Qt, SQL}{}{A program for taking the Myers–Briggs Type Indicator (MBTI) personality test.}

\cventry{Fall 2023}{Text editor}{\href{https://github.com/pr1kol2005/simple-editor}{Github page (clickable)}}{Used tools: Python}{}{A simple text editor designed for use in an internet browser.}


\section{Education}

\cventry{2023 -- present}{Moscow Institute of Physics and Technology}{finished 4th semester bachelor}{Overall GPA 4.70/5, Programming courses GPA 4.81/5}{}{Phystech School of applied Mathematics and Informatics.}

\cventry{2019 -- 2023}{Moscow State School 57}{8-11 grade}{}{GPA 5/5}{Focus on physics and math. Graduated with federal and Moscow gold medals.}


\section{Achievements}

\cvitem{2022 -- 2023}{All-Russian Olympiad for schoolchildren in physics (Final stage participant, top 80 in country); Phystech (MIPT) Olympiad in physics (Gold); Rosatom Olympiad in physics (Silver); Moscow Olympiad for schoolchildren in physics (Silver).}

\cvitem{2021 -- 2022}{Rosatom Olympiad in physics and maths (Gold, Silver); All-Russian Olympiad for schoolchildren in physics (Regional stage prize winner); Phystech (MIPT) Olympiad in physics and maths (Silver, Silver).}

\cvitem{2020 -- 2021}{All-Russian Olympiad for schoolchildren in physics (Regional stage prize winner); Moscow Olympiad for schoolchildren in physics (Silver).}

\cvitem{2019 -- 2020}{International Experimental Physics Olympiad (Bronze); Moscow Olympiad for schoolchildren in physics (Silver).}


\section{Extracurricular activities}

\cventry{2019 -- 2023}{Olympiad Physics Classes}{}{}{}{Theoretical and experimental training for All-Russian Olympiad for schoolchildren in physics, organized by the Moscow City Department of Education.}

\cventry{2020 -- 2022}{Yandex Lyceum}{}{}{}{Python programming classes for high school students. \href{https://academy.yandex.ru/lyceum/}{More info (clickable)}.}

\cventry{2021}{QuSoft Quantum Quest}{}{}{}{An online course on quantum computing for high school students, developed by Michael Walter and Māris Ozols. \href{https://www.quantum-quest.org}{More info (clickable)}.}


\section{Courses taken}

\cvitem{MIPT}{Analytical Geometry; Introduction to Mathematical Analysis; General Physics:~Mechanics; Algebra of Logic, Combinatorics, Graph Theory; Python Practicum; Linear Algebra; Multivariate Analysis, Integrals, and Sequences; General Physics:~Thermodynamics and Molecular Physics; General Physics:~Laboratory Practicum 1~--~2; Fundamentals of Higher Algebra and Coding Theory; Programming Technologies; Multiple Integrals and Field Theory; Fundamentals of Theory of Measure and Probability; Computer Architecture and Operating Systems; Harmonic Analysis; Probability Theory; Computing  Architecture and Assembly Languages; Concurrency; Discrete Structures 1~--~2; Differential Equations 1~--~2; Algorithms and Data Structures 1~--~3; Programming in C++ 1~--~2.}


\end{document}
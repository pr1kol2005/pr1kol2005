\documentclass[a4paper, 12pt, sans]{moderncv}

\usepackage[a4paper, top=0.5cm, bottom=1.5cm, left=1.75cm, right=1.75cm]{geometry}
\usepackage[T1]{fontenc}
\usepackage[utf8]{inputenc}

\usepackage[english]{babel}

\usepackage{mathtext}
\usepackage{mathtools}
\usepackage{mathrsfs}
\usepackage{amsmath,amssymb}
\usepackage{float}

\usepackage{indentfirst}
\usepackage{longtable}


\moderncvstyle{classic}
\moderncvcolor{burgundy}
\setlength{\hintscolumnwidth}{3.0cm}

\firstname{Kirill}
\familyname{Grinko}
\address{Levoberezhnaya 4k2, 18}{125445 Moscow, Russian Federation}
\mobile{+7 916 825 87 57}
\email{k.a.grinko@gmail.com}
\social[telegram]{pr1kol2005}
\social[github]{pr1kol2005}
% \photo[96pt][0.2pt]{pic}

\begin{document}

\makecvtitle


\section{Personal}

\cvitem{Hard skills}{C++, Python, Algorithms and data structures, Assembly x86-64, C, Git, CMake, Bash, Docker, Qt, SFML, Gitlab CI/CD, GoogleTest, LaTeX}

\cvitem{Soft skills}{Hard-working, Quick-learning, Organised, Outgoing and collaborative}

\cvitem{Languages}{English (B2), Russian (native speaker)}

% \cvitem{Sphere of interests}{Mathematics, Game development, Machine learning}

\cvitem{Hobbies}{Calisthenics, skiing, cycling, piano}


\section{Projects}

\cventry{Fall 2024}{Graphing Calculator}{\href{https://github.com/pr1kol2005/graphing-calculator}{Github page (clickable)}}{Used tools: C++, SFML, CMake}{}{Plotter and graphing calculator}

\cventry{Fall 2024}{Algorithms and data structures course homework}{\href{https://github.com/pr1kol2005/algorithms-3rd-term}{Github page (clickable)}}{Used tools: C++}{}{Various algorithms and data structures implementations}

\cventry{Spring - Fall 2024}{C++ course homework}{\href{https://github.com/pr1kol2005/cpp-3rd-term}{Github page (clickable)}}{Used tools: C++}{}{Lots of problems solutions and data structures implementations}

\cventry{Spring 2024}{Box with molecules}{\href{https://github.com/pr1kol2005/box-with-molecules}{Github page (clickable)}}{Used tools: C++, Qt, CMake}{}{Simulation of ideal gas in enclosed space (and a little research about testing the validity of the Maxwell distribution)}

\cventry{Fall 2023}{MBTI test}{\href{https://github.com/pr1kol2005/test-mbti}{Github page (clickable)}}{Used tools: Python, Qt, SQL}{}{A program for completing Myers–Briggs personality test}

\cventry{Fall 2023}{Text editor}{\href{https://github.com/pr1kol2005/simple-editor}{Github page (clickable)}}{Used tools: Python}{}{Simple internet browser text editor}


\section{Education}

\cventry{2023 -- present}{Moscow Institute of Physics and Technology}{finished 1st year bachelor}{Overall GPA 8.35/10, Programming courses GPA 8.43/10}{}{Phystech School of applied Mathematics and Informatics}

\cventry{2019 -- 2023}{Moscow State School 57}{8-11 grade}{}{GPA 5/5}{Focus on physics and math. Graduated with federal and Moscow gold medals}


\section{Courses taken}
\cvitem{MIPT}{Algorithms and Data Structures; Analytical Geometry; Introduction to Calculus; General Physics: Mechanics; Algebra of Logic, Combinatorics, Graph Theory; General Physics: Laboratory Practicum; Python Practicum; Programming in C++; Linear Algebra; Multivariate Calculus, Integrals and Series; General Physics: Thermodynamics and Molecular Physics; Foundations of Higher Algebra and Coding Theory; Programming Technologies}


\section{Achievements}

\cvitem{2022 -- 2023}{All-Russian Olympiad for schoolchildren in physics (Final stage participant, top 80); Phystech (MIPT) Olympiad in physics (Gold); Rosatom Olympiad in physics (Silver); Moscow Olympiad for schoolchildren in physics (Silver)}

\cvitem{2021 -- 2022}{Rosatom Olympiad in physics and maths (Gold, Silver); All-Russian Olympiad for schoolchildren in physics (Regional stage prize winner); Phystech (MIPT) Olympiad in physics and maths (Silver, Silver)}

\cvitem{2020 -- 2021}{All-Russian Olympiad for schoolchildren in physics (Regional stage prize winner); Moscow Olympiad for schoolchildren in physics (Silver)}

\cvitem{2019 -- 2020}{International Experimental Physics Olympiad (Bronze); Moscow Olympiad for schoolchildren in physics (Silver)}


\section{Extracurricular activities}

\cventry{2019 -- 2023}{Olympiad Physics Classes}{}{}{}{Theoretical and experimental training for advanced physics olympiads organised by Moscow city education ministry}

\cventry{2020 -- 2022}{Yandex Lyceum}{}{}{}{Python coding classes for high school students. \href{https://academy.yandex.ru/lyceum/}{More info (clickable)}}

\cventry{2021}{QuSoft Quantum Quest}{}{}{}{Web class for high school students about quantum computing created by Michael Walter and Māris Ozols. \href{https://www.quantum-quest.org}{More info (clickable)}}


\end{document}
